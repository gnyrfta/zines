\documentclass{article}
\usepackage{graphicx}
\usepackage{verbatim}
\usepackage{geometry}
\usepackage{printsudoku}
\usepackage{enumitem,amssymb}
\usepackage{pifont}
\usepackage{fancyhdr}
\usepackage{lastpage}
\newlist{todolist}{itemize}{2}
\setlist[todolist]{label=$\square$}
\newcommand*{\Lpack}[1]{\textsf{#1}}
%\pagenumbering{gobble}
%\fancypagestyle{plain}{%
%\fancyhf{} % clear all header and footer fields
%\fancyfoot[C]{\bfseries \thepage} % except the center
%\renewcommand{\headrulewidth}{0pt}
%\renewcommand{\footrulewidth}{0pt}}
% \pagestyle{empty}
%\renewcommand{\headrulewidth}{0pt}
%\fancyhf{}
% \noversenumbers
\title{Things to do two hours before you go to bed without screens.}
\date{}
\begin{document}
\renewcommand{\thesection}{}\maketitle
\clearpage
\begin{textit}
  \tableofcontents
\end{textit}

\section*{Hi!}
Sleeping is something we all like to do. If youre like me, a lot of the things you enjoy doing during the day involves screens.
But screens are not good to look at before you go to bed - I guess its the blue light, but probably its also the amount of information
that tends to go with it. As almost everybody has heard by now - our minds are not evolved for the amount of information we get today.

There may be research that proves that screens close to bedtime are bad for your sleep, but for me its enough that I have noticed that
it seems to have that effect on me personally. If you feel the same, hopefully this booklet will be helpful! 
This booklet is not intended to be read cover to cover, but rather to be hung on your bulletin-board for some inspiration when you really want
to watch Rick and Morty (or whatever is your poison), but can see nighttime approaching...
\\


\emph{This is useless unless you figure out when you're bedtime is... Personally, I like to sleep 9 hours, but thats up to you. : ) }
\\




\newpage

\section{Sudoku}
The goal of sudoku is to fill each subsquare with each of the numbers 1-9. Every horizontal and vertical row in the large square should also contain each of the numbers 1-9.
\\
Good luck! 


\cluefont{\LARGE}
\cellsize{2.25\baselineskip}

\pagestyle{headings}

\renewcommand*{\puzzlefile}{se5.sud}
\writepuzzle%
{.2...5863}{56.2.3.9.}{.3...7251}%
{..975....}{..6..47.9}{.7..286..}%
{6.58...7.}{8....1..6}{3.7.6..4.}%
[SSBL easy 5]
\vfill
\begin{minipage}{0.95\linewidth}\begin{center}
%SE5 (easy) \\
\sudoku{se5.sud}
\end{center}\end{minipage}

\renewcommand*{\puzzlefile}{se10.sud}
\writepuzzle%
{..6.497..}{.82.1.6..}{79..8.145}%
{649.5.27.}{..7.6..5.}{.3..72.96}%
{.2....81.}{.7..28...}{......5.7}%
%[SSBL easy 10]
\vfill
\begin{minipage}{0.95\linewidth}\begin{center}
\sudoku{se10.sud}
\end{center}\end{minipage}

\renewcommand*{\puzzlefile}{se15.sud}
\writepuzzle%
{4.67.39.2}{..16.84..}{.7...4..1}%
{5...4.21.}{.2.....6.}{.68.7...9}%
{6..5...9.}{..24.93..}{9.53.17.4}%
%[SSBL easy 15]
\vfill
\begin{minipage}{0.95\linewidth}\begin{center}
%SE15 (easy) \\
\sudoku{se15.sud}
\end{center}\end{minipage}

\renewcommand*{\puzzlefile}{se20.sud}
\writepuzzle%
{36.9..5..}{..54.8.2.}{.....78.9}%
{2....3.68}{.83.9....}{.4.7..2.3}%
{6.71..3..}{13.8.59.2}{..9.7..56}%
%[SSBL easy 20]
\vfill
\begin{minipage}{0.95\linewidth}\begin{center}
%SE20 (easy) \\
\sudoku{se20.sud}
\end{center}\end{minipage}

\renewcommand*{\puzzlefile}{se25.sud}
\writepuzzle%
{....63.5.}{6.5.1.839}{..1...2..}%
{823...7..}{.5...7...}{.64.9...1}%
{4.25796.3}{.9.1.65.8}{.1...89.7}%
[SSBL easy 25]
\vfill
\begin{minipage}{0.95\linewidth}\begin{center}
%SE25 (easy) \\
\sudoku{se25.sud}
\end{center}\end{minipage}

\renewcommand*{\puzzlefile}{se30.sud}
\writepuzzle%
{.3.12....}{1.87..4.6}{..98....1}%
{5....7.3.}{.73.9.2.4}{.92..65..}%
{.4...8.5.}{3...7...8}{2..4.596.}%
%[SSBL easy 30]
\vfill
\begin{minipage}{0.95\linewidth}\begin{center}
%SE30 (easy) \\
\sudoku{se30.sud}
\end{center}\end{minipage}

\begin{comment}
\renewcommand*{\puzzlefile}{tg5.sud}
\writepuzzle%
{.6.1.9.8.}{.19...74.}{2.......1}%
{..79.26..}{.3.....2.}{..14.35..}%
{1.......5}{.75...23.}{.9.5.6.7.}%
[DT2 gentle 5]
\vfill
\begin{minipage}{0.95\linewidth}\begin{center}
%TG5 (gentle) \\
\sudoku{tg5.sud}
\end{center}\end{minipage}

\renewcommand*{\puzzlefile}{tg10.sud}
\writepuzzle%
{....96..4}{..1.....2}{56...8...}%
{2.8....9.}{9.63.52.7}{.3....4.6}%
{...9...58}{7.....9..}{8..45....}%
[DT2 gentle 10]
\vfill
\begin{minipage}{0.95\linewidth}\begin{center}
%TG10 (gentle) \\
\sudoku{tg10.sud}
\end{center}\end{minipage}

\end{comment}

\renewcommand*{\puzzlefile}{tg15.sud}
\writepuzzle%
{...1.8...}{.57...18.}{98.....26}%
{..67419..}{.........}{..42536..}%
{72.....94}{.19...73.}{...3.9...}%
[DT2 gentle 15]
\vfill
\begin{minipage}{0.95\linewidth}\begin{center}
%TG15 (gentle) \\
\sudoku{tg15.sud}
\end{center}\end{minipage}

\renewcommand*{\puzzlefile}{tg20.sud}
\writepuzzle%
{...7.9...}{.8.3.6.5.}{7.9...6.8}%
{..42.18..}{.6.....4.}{..35.41..}%
{9.2...5.7}{.4.8.7.1.}{...1.2...}%
[DT2 gentle 20]
\vfill
\begin{minipage}{0.95\linewidth}\begin{center}
%TG20 (gentle) \\
\sudoku{tg20.sud}
\end{center}\end{minipage}
\vspace{1cm}

If you enjoyed solving these - there are plenty of sudoku files on the internet, to be downloaded and printed well before bedtime. : )
\newpage
\section{Mandalas}
 \pagestyle{empty}

Here are 5 mandalas you can colour.
\vspace{3cm}
\setlength\oddsidemargin{\dimexpr(\paperwidth-\textwidth)/2 - 1in\relax}
\setlength\evensidemargin{\oddsidemargin}

\includegraphics[width=15cm]{marmot}
\\
\includegraphics[width=15cm]{mandala2}
\\
\includegraphics[width=15cm]{mandala3}
\\
\includegraphics[width=15cm]{mandala4}
\\
\includegraphics[width=15cm]{mandala5}
%\includegraphics[width=15cm, height=15cm]{16724}
%\includegraphics[width=15cm, height=15cm]{16724}
\\
If you enjoyed colouring these - there are plenty of mandala files on the internet, to be downloaded and printed well before bedtime. : )
\newpage
\section{Inception tic-tac-toe}
If you have a companion in your non-screen-ness you can play inception tic-tac-toe!
Its just like normal tic-tac-toe - every turn you put a x or o in one of the boxes - but you don't win until you have three won sub-boxes in a row.
Like in this pretty example:
\\
\includegraphics[width=4cm]{tic_tac_toe}

\renewcommand*{\puzzlefile}{se5.sud}
\writepuzzle%
{.........}{.........}{.........}%
{.........}{.........}{.........}%
{.........}{.........}{.........}%
[SSBL easy 5]
\vfill
\begin{minipage}{0.95\linewidth}\begin{center}
%SE5 (easy) \\
\sudoku{se5.sud}
\end{center}\end{minipage}

\renewcommand*{\puzzlefile}{se5.sud}
\writepuzzle%
{.........}{.........}{.........}%
{.........}{.........}{.........}%
{.........}{.........}{.........}%
[SSBL easy 5]
\vfill
\begin{minipage}{0.95\linewidth}\begin{center}
%SE5 (easy) \\
\sudoku{se5.sud}
\end{center}\end{minipage}

\renewcommand*{\puzzlefile}{se5.sud}
\writepuzzle%
{.........}{.........}{.........}%
{.........}{.........}{.........}%
{.........}{.........}{.........}%
[SSBL easy 5]
\vfill
\begin{minipage}{0.95\linewidth}\begin{center}
%SE5 (easy) \\
\sudoku{se5.sud}
\end{center}\end{minipage}

\renewcommand*{\puzzlefile}{se5.sud}
\writepuzzle%
{.........}{.........}{.........}%
{.........}{.........}{.........}%
{.........}{.........}{.........}%
[SSBL easy 5]
\vfill
\begin{minipage}{0.95\linewidth}\begin{center}
%SE5 (easy) \\
\sudoku{se5.sud}
\end{center}\end{minipage}

I think you will manage to make your own if the ready-made boards run out. ; )
\newpage
\section{Suggestion-lists}


One suggested use for these: Rip one out per week and put up on your refrigerator. Check the ones you've done.
Obviously its quite personal what shit you find fun to do, so there are five blank checkboxes where you can add your own stuff (if you want - obviously). 


\noindent\dotfill
\ding{33}\dotfill
\raisebox{-0.25\baselineskip}{\ding{34}}\dotfill
\raisebox{-0.50\baselineskip}{\ding{35}}\dotfill



\begin{itemize}
  \item Idea-list for things you can do without screens.
  \begin{todolist}
    \item make bread (basic bread recipe further down)
    \item take a long walk and then take a bath and read a good book (list of good books to be inserted).
    \item write reflections on the day on paper.
    \item drink hippie tea (ginger and honey).
    \item Do yoga without following a video - this means you probably have to do the same routine every time. I can't do this yet - hope you can!
    \item Play an instrument
    \item Actually listen to a full album.
    \item Meditate, accept yourself for who you are, chill with your faults.
    \item Talk to a plant about the videos you would like to watch. 
    \item Talk to a human (warning - they talk back!).
    \item Make food for the next day and enjoy having lots of time to do it.     
    \item \vspace {1cm}
    \item \vspace {1cm}
    \item \vspace {1cm}
    \item \vspace {1cm}
    \item \vspace {1cm}     
    \end{todolist}
\end{itemize}

\newpage

\noindent\dotfill
\ding{33}\dotfill
\raisebox{-0.25\baselineskip}{\ding{34}}\dotfill
\raisebox{-0.50\baselineskip}{\ding{35}}\dotfill



\begin{itemize}
  \item Idea-list for things you can do without screens.
  \begin{todolist}
    \item make bread (basic bread recipe further down)
    \item take a long walk and then take a bath and read a good book (list of good books to be inserted).
    \item write reflections on the day on paper.
    \item drink hippie tea (ginger and honey).
    \item Do yoga without following a video - this means you probably have to do the same routine every time. I can't do this yet - hope you can!
    \item Play an instrument
    \item Actually listen to a full album.
    \item Meditate, accept yourself for who you are, chill with your faults.
    \item Talk to a plant about the videos you would like to watch. 
    \item Talk to a human (warning - they talk back!).
    \item Make food for the next day and enjoy having lots of time to do it.     
    \item \vspace {1cm}
    \item \vspace {1cm}
    \item \vspace {1cm}
    \item \vspace {1cm}
    \item \vspace {1cm}     
    \end{todolist}
\end{itemize}
\newpage

\noindent\dotfill
\ding{33}\dotfill
\raisebox{-0.25\baselineskip}{\ding{34}}\dotfill
\raisebox{-0.50\baselineskip}{\ding{35}}\dotfill



\begin{itemize}
  \item Idea-list for things you can do without screens.
  \begin{todolist}
    \item make bread (basic bread recipe further down)
    \item take a long walk and then take a bath and read a good book (list of good books to be inserted).
    \item write reflections on the day on paper.
    \item drink hippie tea (ginger and honey).
    \item Do yoga without following a video - this means you probably have to do the same routine every time. I can't do this yet - hope you can!
    \item Play an instrument
    \item Actually listen to a full album.
    \item Meditate, accept yourself for who you are, chill with your faults.
    \item Talk to a plant about the videos you would like to watch. 
    \item Talk to a human (warning - they talk back!).
    \item Make food for the next day and enjoy having lots of time to do it.     
    \item \vspace {1cm}
    \item \vspace {1cm}
    \item \vspace {1cm}
    \item \vspace {1cm}
    \item \vspace {1cm}     
    \end{todolist}
\end{itemize}
\newpage

\noindent\dotfill
\ding{33}\dotfill
\raisebox{-0.25\baselineskip}{\ding{34}}\dotfill
\raisebox{-0.50\baselineskip}{\ding{35}}\dotfill



\begin{itemize}
  \item Idea-list for things you can do without screens.
  \begin{todolist}
    \item make bread (basic bread recipe further down)
    \item take a long walk and then take a bath and read a good book (list of good books to be inserted).
    \item write reflections on the day on paper.
    \item drink hippie tea (ginger and honey).
    \item Do yoga without following a video - this means you probably have to do the same routine every time. I can't do this yet - hope you can!
    \item Play an instrument
    \item Actually listen to a full album.
    \item Meditate, accept yourself for who you are, chill with your faults.
    \item Talk to a plant about the videos you would like to watch. 
    \item Talk to a human (warning - they talk back!).
    \item Make food for the next day and enjoy having lots of time to do it.     
    \item \vspace {1cm}
    \item \vspace {1cm}
    \item \vspace {1cm}
    \item \vspace {1cm}
    \item \vspace {1cm}     
    \end{todolist}
\end{itemize}
\newpage

\noindent\dotfill
\ding{33}\dotfill
\raisebox{-0.25\baselineskip}{\ding{34}}\dotfill
\raisebox{-0.50\baselineskip}{\ding{35}}\dotfill



\begin{itemize}
  \item Idea-list for things you can do without screens.
  \begin{todolist}
    \item make bread (basic bread recipe further down)
    \item take a long walk and then take a bath and read a good book (list of good books to be inserted).
    \item write reflections on the day on paper.
    \item drink hippie tea (ginger and honey).
    \item Do yoga without following a video - this means you probably have to do the same routine every time. I can't do this yet - hope you can!
    \item Play an instrument
    \item Actually listen to a full album.
    \item Meditate, accept yourself for who you are, chill with your faults.
    \item Talk to a plant about the videos you would like to watch. 
    \item Talk to a human (warning - they talk back!).
    \item Make food for the next day and enjoy having lots of time to do it.     
    \item \vspace {1cm}
    \item \vspace {1cm}
    \item \vspace {1cm}
    \item \vspace {1cm}
    \item \vspace {1cm}     
    \end{todolist}
\end{itemize}
\newpage
\section{A bread recipe and a totally objective list of great books}

\subsection*{Bread Recipe}
This is a basic bread recipe:
\begin{itemize}
  \item 500 g of wheat flour
  \item 325 g of water
  \item 10 g of salt (about 1.5 tsp)
  \item 3 g of fresh yeast.
\end{itemize}
 \begin{enumerate}
    \item Mix all ingredients together until you have a smooth silky dough    
    \item Wait until dough doubled in size (also known as bulk fermentation). This takes on average ~8 hours
    \item Shape a dough ball
    \item Place in baneton or in a floured bowl
    \item Wait until almost doubled in size (also known as proofing). This takes on
      average 2 hours at room temperature or 24 hours in the fridge
    \item Bake 20 minutes in preheated oven at 230°C with a bowl of water
    \item Remove bowl of water after 20 minutes, bake another 20 minutes
    \item Baking is finished when bread has desired brown color
\end{enumerate}
    Now I followed this recipe, \textbf{but ended up using 25 g of yeast and quite a lot more flour.} But if you start with this recipe and just add more flower and probably also yeast maybe it works for you too! No shade on the recipe-creator (user hendricius on github - the-bread-code), he was probably using American ingredients
    which are probably a bit different than Swedish ingredients. The bread turned out good in the end! 
    The ingredients are cheap, too.
    
\newpage
\subsection*{A totally objective list of great books}
Ok, so of course a book-list is one of the most subjective things you can make. But since this is partly self-therapy I'm inserting it anyway.
This is based on a list made by me for an improvised book café a summer ago and its the books i wanted to find complemented with suggestions from friends.
Remember to pick up a book somewhere before bedtime. The local library is your friend. : ) 

\textbf{Authors, mainly of books}
\begin{itemize}
\item Ursula Le Guin, 
\item Michael Ende, 
\item Tove Jansson, 
\item Dostoyevski, 
\item Leo Tolstoy, 
\item Douglas Adams
\item Shakespeare,
\item Kafka,
\item Neil Gaiman,
\item Terry Pratchett, 
\item Mythology books illustrated by Michael Foreman, 
\item Umberto Eco, 
\item Edgar Allan Poe, 
\item Stephen King,
\item JD Salinger, 
\item Niklas natt och dag - 1793
\item Alexander Solsjenitsyn
\item George Orwell, 
\item Isaac Asimov, 
\item Ray Bradbury,
\item Sophocles, maybe. I guess. 
\item Vernor Vinge, 
\item Haruki Murakami, 
\item Falstaff Fakir
\item Selma Lagerlöf
\item Cixin Liu
\end{itemize}
\textbf{Authors, mainly of poems}
\\
\begin{itemize}
\item Baudelaire
\item Bashao or whatever his name was, Japan - haven't read much
\item Issa, haven't read much
\item Nils Ferlin,
\item Fröding, 
\item Majakovskij
\end{itemize}
\textbf{Titles}
\\
\begin{itemize}
\item TH White - Arthur, 
\item Bhagavad Gita,
\item Narnia, 
\item The first harry potter book, 
\item On the road, Jack Kerouack, 
\item 5 Rings or something, some kind of welsh story, 
\item The moon of Gomrath, 
\item The lord of the Flies, 
\item Beowulf, the translation by Seamus Heaney, 
\item The Koran 
\item The Bible
\item Faust 
\item Odysseus, by James Joyce,
\end{itemize}

Credits -
\\
The first mandala was created by LaTeX user user121799.
\\
The second to last mandala was created from here: https://grundschulblogs.blogspot.com/2020/07/inspirierend-mandala-1-klasse-religion.html (you can look at it when its not two hours before bedtime).
\\
The last mandala has a to me unknown creator - if you know who made it let me know so I can credit that person. : )
\\
The other two mandalas are in the public domain.
\\
Mandala on the very last page is made with code from LaTeX user David Carlisle
\\
The bread recipe is, as mentioned, from The Breadcode on Github, a repository created by Github user hendricius.
\\
The sudoku puzzles included here were created by Peter Wilson, who also created a LaTeX package for generating and solving sudoku - sudokubundle (available at CTAN).
Booklet created using the free software LaTeX, emacs and Gimp and printed at our awesome hacklab Linkping in Linköping. 
\\
This booklet was created by David Jacobsson - you can e-mail me at david@gnyrftacode.se two hours before your bedtime. 

\newpage
% Mandala
\begin{center}
\begin{picture}(40,40)
\put(00,00){\framebox(40,40){}}\put(20,20){\circle{30}}
\put(20,20){\circle*{2}}            \linethickness{2pt}
\put(15,01){\line(1,0){10}}\put(20,00){\line(00,01){4}}
\put(15,39){\line(1,0){10}}\put(20,40){\line(00,-1){4}}
\put(01,15){\line(0,1){10}}\put(00,20){\line(01,00){4}}
\put(39,15){\line(0,1){10}}\put(40,20){\line(-1,00){4}}
\end{picture}
\end{center}
\end{document}
%%% Local Variarbles:
%%% mode: latex
%%% TeX-master: t
%%% End:
 
%%% Local Variables:
%%% mode: latex
%%% TeX-master: t
%%% End:
